\documentclass[11pt,a4paper, twocolumn]{article}
\usepackage[utf8]{inputenc}
\usepackage{amsmath}
\usepackage{amsfonts}
\usepackage{amssymb}
\usepackage{graphicx}
\usepackage[left=2cm,right=2cm,top=2cm,bottom=2cm]{geometry}
\usepackage{lipsum}
\usepackage{graphicx}
\usepackage{float}
\usepackage{enumerate}
\usepackage{color}
\usepackage{soul}

\newcommand{\thev}{Th\'{e}venin Equivalent}

\setlength\columnsep{40pt}

\title{\textbf{Week 5: EGB120 lecture notes }}
\author{By Pilote Muhoza}
\date{\today}


\begin{document}
\maketitle
\section{Non-Ideal Source}
\begin{itemize}
\item Real sources often have many limitation in the term of voltage and current delivery.
\item The most commonly modelled, and most useful for a lenear circuit theory, is some form of resistance associated with the source
\end{itemize}

%% Add figure

\section {Th\'{e}venin Equivalent}
\begin{itemize}

\item The Th\'{e} venin equivalent circuit is a voltage source with  series resistance.
\item characterised by the three related parameters: $v_{Th}, R_{Th}, $ and $i_{sc}$.
\end{itemize}

%% figure

	\subsection{Formula}
	\begin{figure}[!h]
		\begin{align}
	i_{scc}=\frac{v_{Th}}{R_{Th}}
	\end{align}
	\caption{Th\'{e}venin formula}
	\end{figure}

	
\begin{figure}[!h]
	\centering
	\includegraphics[width=10 cm]{thev}
	\caption{{Thev calculations}}
\end{figure}
\newpage
\section{Norton Equivalent}
\begin{itemize}
\item You can equally use a current source with a resistor in parallel to get exactly the same properties.
\item This is called equivalent circuit.

	\end{itemize}
	\begin{figure}[!h]
	\centering
	\includegraphics[scale=0.5]{nor}
	\caption{Nortorn equivalent}
	\end{figure}
\section {\textbf{\thev $<=>$ Nortorn Equivalent}}

	\begin{itemize}
	\item You can substitute a \thev circuit for a Norton equivalent circuit and vice versa
	\item	Calculating equivalent component values required no extra info:
	
	\end{itemize}
	\begin{figure}[!h]
	\includegraphics[scale=0.4]{new}
	\caption{\thev and Norton relationship}
	\end{figure}
	

		%%page 2

\section{Superposition}

	\begin{enumerate}
	\item 	Superposition is a principal of linear systems.
	\item We can use it to simplify circuit analysis by noting that we can treat each source independently. 
	\item Circuits can be simplified by having only one source active at a time, changing all the other sources to zero.
	\item Net effect on the coltage or current can be found by the summing components due to individual sources.
	\end{enumerate}
	\subsection{Conditions}
	\begin{enumerate}
	
	\item	For a \textcolor{blue}{\textbf{current source}}, changing to zero means replacing it with an \textbf{open circuit} \hl{(zero current, whatever voltage)}.
	\item For a \textcolor{red}{\textbf{voltage source}} changing to zero means replacing it with a \textbf{short circuit} \hl{(zero voltage, whatever current)}.
	\item Work out total voltage or current from the sum of the individual contributions.
	\end{enumerate}
	\subsection{Example calculations}
	
	\begin{figure}[!h]
	\includegraphics[scale=0.3]{ex}
	\caption{Step (1)}
	\includegraphics[scale=0.3]{ex1}\caption{Step (2)}
	
	
	\end{figure}
	
	\begin{figure}[H]
	\includegraphics[scale=0.3]{ex2}
	\caption{Step (3)}
	\includegraphics[scale=0.3]{ex3}
	\end{figure}

	
\section{Th\'{e}venin's Theorem}

	\begin{figure}[H]
	\includegraphics[scale=0.3]{re}
	\caption{Step (1)}
	\end{figure}
	
	\begin{figure}[H]
	
	\includegraphics[scale=0.3]{re1}
	\caption{ Step(2)}
	\end{figure}
	
	\begin{figure}[H]
	\includegraphics[scale=0.3]{re2}
	\caption{Step (3)}
	\end{figure}

\section{Northon Theoren}
	\begin{figure}[H]
	\centering
	\includegraphics[scale=0.3]{n}
	\caption{Sample calculations of Notorn Theorem}
	
	\end{figure}

	
	
	
	
	
	
	
\end{document}